%
% 6.857 homework template
%
% NOTE:
% Be sure to define your team members with the \team command
% Be sure to define the problem set with the \ps command
% Be sure to use the \answer command for each of your answers 
\documentclass[11pt]{article}

\newcommand{\team}{ Victor Williamson \\ Victor Chan \\ David Feldman }
\newcommand{\ps}{ Problem Set 2 }

%\pagestyle{headings}
\usepackage[dvips]{graphics,color}
\usepackage{amsfonts}
\usepackage{amssymb}
\usepackage{amsmath}
\usepackage{latexsym}
\usepackage{verbatim}
\setlength{\parskip}{1pc}
\setlength{\parindent}{0pt}
\setlength{\topmargin}{-3pc}
\setlength{\textheight}{9.5in}
\setlength{\oddsidemargin}{0pc}
\setlength{\evensidemargin}{0pc}
\setlength{\textwidth}{6.5in}

\newcommand{\answer}[1]{
\newpage
\noindent
\framebox{
	\vbox{
		6.857 Homework \hfill {\bf \ps} \hfill \# #1  \\ 
		\team \hfill \today
	}
}
\bigskip

}


\begin{document}

\answer{2-1 - MAC Attack}

a) This MAC procedure is not a good one. It is possible for an adversary to recover K using merely the original message M and MAC(K,M), both of which are sent in the clear
and are readily available to the adversary.

During creation of the MAC we let:
\newline
\begin{equation*}
V_0 = K
\end{equation*}
and proceed with 
\begin{equation*}
V_i = encryptAES(M_i,V_{i-1})
\end{equation*}
for each i = 1,2,...,n
\begin{equation}
MAC(K,M) = V_n
\end{equation}

It is very simple to reverse this procedure. We would again pad M and break it into n 128-bit blocks. We then use AES decryption as follows:

\begin{equation*}
V_{i-1} = decryptAES(M_(i), V_i)
\end{equation*}
for all i = n, n-1,...,1,0 where
\begin{equation*}
V_n = MAC(K,M)
\end{equation*}
as received from the sender, and
\begin{equation*}
K = V_0
\end{equation*} 

Once K is obtained then the attacker could write fake messages and pass them off as authentic by using the MAC creation procedure described above, since the key K is now known. Essentially, the problem occurred with this MAC procedure because it reversed the position of the key and message block within AES encryption. This change made it possible to recover the  secret key, which would have been used between the users to validate authenticity.

Note that
\begin{equation*}
xxxAES(K,B)
\end{equation*}
denotes the encrpytion or decryption of a 128-bit plaintext block B with a 128-bit key K using the AES algorithm.

b) Adding a checksum to this MAC will not make it any better. This adds no protection to the MAC because the XOR of the message blocks does not prevent the key from being retrieved. Since an attacker is able to retrieve K, they will still be able to create fake messages and the correct MAC's these messages. It would be trivial to add on an additional "correct" checksum of the fake message.  


\begin{comment}
Different font styles: 

Several ways to do \textit{italics}, {\it italics}, \emph{italics}.

Several ways to do \textbf{bold}, {\bf bold}.

Other fonts: \textsc{Blah}.

{\tiny tiny}, {\small small}, {\normalsize normalsize}, {\Large Large}. {\LARGE
LARGE}. {\huge huge}, {\Huge Huge}.

F\"{o}\.{r}\'{e}\r{i}\^{g}\u{n} Accents.
\end{comment}
\answer{2-2 - Truncated-Key AES}

\begin{comment}
Here's an example of how to do mathematical symbols in an Equation
array:

\begin{eqnarray*}
L & = & \mbox{Total lines of code.} \\
\rho & = & \mbox{Rate of line replacement.} \\
\delta & = & \mbox{Density of bugs in replaced line.} \\
V(t) & = & \mbox{Total number of vulnerabilities at time  } t. \\
\gamma & = & \mbox{Debugger inspection rate in the open source model.}\\
\hat{\gamma} & = & \mbox{Debugger inspection rate in the proprietary model .}\\
\beta & = & \mbox{Adversarial inspection rate in the open source model.}\\
\hat{\beta} & = & \mbox{Adversarial inspection rate in the proprietary model.}\\
\end{eqnarray*}

Or you can put symbols inline like this: $\rho$, $\gamma$, and
$\frac{\alpha}{\beta}$.

Some symbols that may come in handy someday: $\Theta$, $\oplus$, $\cdot$,
$\ldots$, $\rightarrow$, $\leq$, $\geq$, $m^e \bmod p$,  $\mathbb{N}$, $\mathbb{Z}$.

Superscripts and subscripts: $x^{super}_{sub}$.

Superscripts and subscripts in an equation environment: 
\begin{equation}
x^{super}_{sub}
\end{equation}
\end{comment}

\answer{2-3 - Ben's Block Cipher}
\begin{comment}
\begin{enumerate}
\item An
\item ordered
\item list
\end{enumerate}

\begin{itemize}
\item An
\item unordered
\item list
\end{itemize}
\end{comment}
\end{document}


